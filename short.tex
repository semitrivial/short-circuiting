\documentclass[runningheads]{llncs}
\usepackage{amsfonts}
\usepackage{amsmath}
\usepackage{mathrsfs}
% \smartqed  % flush right qed marks, e.g. at end of proof
% \usepackage{graphicx}

\newcommand{\myclaim}[1]{\textbf{Claim #1:}}
\def\myulcorner{\mathord{\ulcorner}}
\def\myurcorner{\mathord{\urcorner}}

\pagenumbering{gobble}

\begin{document}

\title{Short-circuiting the definition of mathematical knowledge for an AGI
%\thanks{}
}

%\titlerunning{Short-circuiting the definition of knowledge}

\author{Samuel Allen
Alexander\inst{1}\orcidID{0000-0002-7930-110X}}

\institute{The U.S.\ Securities and Exchange Commission
\email{samuelallenalexander@gmail.com}
\url{https://philpeople.org/profiles/samuel-alexander/publications}}


\maketitle

\begin{abstract}
Fill this in.
\keywords{AGI \and machine knowledge \and Moore's paradox}
\end{abstract}

\section{Introduction}

It is difficult to define knowledge, or what it means to know something.
In Plato's dialogs, again and again Socrates asks people to define
knowledge\footnote{Perhaps the best example being in the \emph{Theaetetus}
\cite{theaetetus}.}, and no-one ever succeeds. Neither have philosophers
reached consensus even in our own era (see \cite{sep-knowledge-analysis}).

At the same time, the problem is often brushed aside as something only
philosophers care about: pragmatists rarely spend time
on this sort of debate. One exceptional area where the question becomes
important even to pragmatists is in the area of Artificial General Intelligence
(AGI). In AGI research, these old philosophical conundrums rear their ugly
once more.

In this paper, we narrow down the question ``what is knowledge'' and offer
a simple answer within that narrow context:
we propose a definition of what it means for a suitably idealized AGI to know
a mathematical sentence\footnote{By a \emph{sentence}, we mean a formula with
no free variables. Thus, $x^2>0$ is not a sentence, but
$\forall x (x^2>0)$.}. Our proposed definition is short and
sweet: we say that
an AGI knows a mathematical sentence (in some standard mathematical language $\mathscr L$)
if and only if that sentence would be among the sentences which that AGI would
enumerate if that AGI were commanded:
``Enumerate, in the language $\mathscr L$, every mathematical sentence which
is both expressible in $\mathscr L$ and part of your own mathematical knowledge.''

Our proposed definition is not directly intended
for methodological purposes---it would not
directly be helpful in the quest to construct an AGI. Instead, it is intended for
theoretical purposes of understanding the properties of AGI (some of which
we survey in Section \ref{appsection}). We hope that a better
understanding of theoretical properties of AGIs will indirectly help in the eventual
creation of same.

The structure of this paper is as follows.
\begin{itemize}
  \item In Section \ref{agisection} we discuss the AGIs whose knowledge we are
  attempting to propose a definition for.
  \item In Section \ref{mainsection} we propose a knowledge definition for
  AGIs for sentences in some standard mathematical language.
  \item In Section \ref{quantifiedsection} we extend our knowledge definition
  to formulas with free variables (when the intended interpretation for the
  language in question is arithmetical).
  \item In Section \ref{appsection} we survey some of the applications of
  this proposed knowledge definition.
  \item In Section \ref{conclusionsection} we summarize and make concluding remarks.
\end{itemize}

\section{Idealized AGIs}
\label{agisection}



In this paper, we approach AGI using what
Goertzel \cite{goertzel2014artificial} calls
the Universalist Approach:
we adopt ``...an idealized case of AGI, similar to
assumptions like the frictionless plane in physics'', hoping that by
understanding this ``simplified special
case, we can use the understanding we've gained to address more realistic
cases.'' We assume the AGI is designed in such a way as to be obey arbitrary
English commands, and is capable (if so commanded) of outputting data in
any desired format, provided that format can be expressed (say) as a computer
file or as a string of unicode characters. Below, we will state some additional
idealized assumptions, but first we will need a preliminary definition.

\begin{definition}
\label{stdmathematicallanguagedefn}
By a \emph{standard mathematical language}, we mean a mathematical language for which an
intended interpretation is implicitly understood.
\end{definition}

For example, the
langauge of Peano Arithmetic is a standard mathematical language, the obvious
intended interpretation being the model with universe $\mathbb N$ which
interprets the arithmetical symbols of Peano Arithmetic in the usual ways.
Presumably an AGI suitably familiar with the mathematical literature should
be aware of this intended interpretation for Peano Arithmetic.

Armed with Definition \ref{stdmathematicallanguagedefn}, we are ready to articulate
two additional assumptions which we make about an AGI $X$
(here $\mathscr L$ is a standard mathematical language):
\begin{itemize}
  \item (Truthfulness) Whatever $\mathscr L$-sentences $X$ enumerates
  when commanded to enumerate the $\mathscr L$-sentences that $X$ knows,
  all such sentences are true (in the intended model).
  \item (Obedience) When commanded to enumerate the $\mathscr L$-sentences that $X$ knows,
  $X$ will obey the command: $X$ will enumerate exactly the
  $\mathscr L$-sentences that $X$ knows (according to $X$'s own internal definition
  of knowledge, whatever that may be).
\end{itemize}

Note that in our \emph{Obedience} assumption, we do not require the AGI to use
the same definition of knowledge that we are proposing in this paper. We allow
the AGI to have whatever definition of knowledge it likes.

The \emph{Truthfulness} and \emph{Obedience} assumptions would be inappropriate
for human agents $X$, who make mathematical
mistakes, and who tend to hold unfounded beliefs and claim them as knowledge,
and who might tend to resist tedious tasks such as enumerating endless lists of
mathematical sentences. But arguably these assumptions are plausible for a
sufficiently idealized AGI. Such an AGI can presumably perform calculations with
no risk of mechanical error. Such an AGI is presumably free of the human
psychological quirks which cause humans to cling to unfounded beliefs (especially
since we are only considering mathematical statements, not contingent on
the physical world). And such an AGI should have unlimited patience and have no
problem tediously enumerating mathematical sentences as long as memory-banks
and electricity are available.


\section{An elegant definition of mathematical knowledge}
\label{mainsection}

If $X$ is an idealized AGI satisfying the assumptions of \emph{Truthfulness}
and \emph{Obedience} from the previous section, we define the mathematical
knowledge of $X$ (as far as sentences go)
as follows (where $\mathscr L$ denotes a standard mathematical
language).

\begin{definition}
\label{maindef}
  For any $\mathscr L$-sentence $\phi$, we say that $X$ knows $\phi$ if and only
  if $X$ would eventually list $\phi$ among the $\mathscr L$-sentences which $X$
  would list if $X$ were commanded:
  ``Enumerate, in the language $\mathscr L$, every mathematical sentence which
  is both expressible in $\mathscr L$ and part of your own mathematical knowledge.''
\end{definition}

One of the strengths of Definition \ref{maindef} is that it is uniform across
different AGIs: many different AGIs might internally operate based on different
definitions of knowledge, but Definition \ref{maindef} works equally well for
all these different AGIs irrespective of those different internal knowledge
definitions\footnote{This is reminiscent of Elton's proposal that instead of
trying to interpret an AI's outputs by focusing on specific low-level details
of a neural network, we should instead let the AI explain itself \cite{elton}.}.

Although Definition \ref{maindef} may differ significantly from a particular AGI
$X$'s own internal definition of knowledge, the following theorem states that
materially the two definitions have the same result.

\begin{theorem}
\label{sentenceequivalence}
  Suppose $X$ is an AGI satisfying Truthfulness and Obedience, and let $\mathscr L$
  be a standard mathematical language. For any $\mathscr L$-sentence $\phi$, the following
  are equivalent:
  \begin{enumerate}
    \item $X$ is
    considered to know $\phi$ (based on Definition \ref{maindef}).
    \item
    $X$ is considered to know $\phi$ (based on $X$'s own inernal definition of
    knowledge).
  \end{enumerate}
\end{theorem}

\begin{proof}
  By Definition \ref{maindef}, (1) is equivalent to the statement that $X$ would
  include $\phi$ in the list which $X$ would output if $X$ were commanded to output
  all the $\mathscr L$-sentences that $X$ knows. By Obedience, $X$ would output
  $\phi$ in that list if and only if (2).
\end{proof}

\subsection{Languages with Knowledge Operators}

Definition \ref{maindef} is particularly interesting when $\mathscr L$ itself
contains an operator for the agent's knowledge. An example of such a language would be
the language of Epistemic Arithmetic from \cite{shapiro}, which consists of the
language of Peano Arithmetic with the addition of an operator $K$ for knowledge:
$K(1+1=2)$ should be read as something like
``I know $1+1=2$'' or ``the knower knows $1+1=2$''. In the context of this paper,
if $\mathscr L_0$ is a standard mathematical language, and if $\mathscr L$ is obtained
from $\mathscr L_0$ by the addition of a knowledge operator $K$, then we also
consider $\mathscr L$ to be a standard mathematical language. The intended model
of $\mathscr L$ shall have the same universe and interpretation of
$\mathscr L_0$-symbols as the intended model of $\mathscr L_0$. As for $K$,
the intended interpretation (by an AGI $X$) of a formula $K(\phi)$ shall be
that $X$ knows $\phi$ (according to the AGI's internal definition of knowledge).

\begin{example}
Applying Definition \ref{maindef} to the language of Epistemic Arithmetic,
we consider an AGI $X$ to know $K(1+1=2)$ if and only if that AGI would output
$K(1+1=2)$ when commanded to output all sentences that $X$ knows in the language of
Epistemic Arithmetic. By the Obedience assumption and the intended interpretation of
Epistemic Arithmetic, $X$ would (when so commanded)
output $K(1+1=2)$ if and only if $X$ knows that he knows $1+1=2$.
\end{example}

\section{Quantified Modal Logic}
\label{quantifiedsection}

\section{Applications}
\label{appsection}

\section{Conclusion}
\label{conclusionsection}

\bibliographystyle{splncs04}
\bibliography{short}

\end{document}