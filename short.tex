\documentclass[runningheads]{llncs}
\usepackage{amsfonts}
\usepackage{amsmath}
\usepackage{mathrsfs}
% \smartqed  % flush right qed marks, e.g. at end of proof
% \usepackage{graphicx}

\newcommand{\myclaim}[1]{\textbf{Claim #1:}}
\def\myulcorner{\mathord{\ulcorner}}
\def\myurcorner{\mathord{\urcorner}}

\pagenumbering{gobble}

\begin{document}

\title{Short-circuiting the definition of mathematical knowledge for an AGI
%\thanks{}
}

%\titlerunning{Short-circuiting the definition of knowledge}

\author{Samuel Allen
Alexander\inst{1}\orcidID{0000-0002-7930-110X}}

\institute{The U.S.\ Securities and Exchange Commission
\email{samuelallenalexander@gmail.com}
\url{https://philpeople.org/profiles/samuel-alexander/publications}}


\maketitle

\begin{abstract}
We propose that, for the purpose of studying theoretical properties of
the knowledge of an agent with Artificial General Intelligence (AGI),
a pragmatic way to define such an agent's knowledge is as follows.
We declare the AGI to know a certain statement
(in a given mathematical language) if and only if, when commanded
to enumerate all the statements that it knows in that language
(with whatever internal knowledge definition it has), said AGI
would include said statement in the resulting enumeration.
This elegantly solves the problem that different AGIs may have
very different internal knowledge definitions and yet we want
to be able to study knowledge of AGIs in general, without having
to study different AGIs separately just because they have separate
internal knowledge definitions. Finally, we suggest how this
definition of AGI knowledge can be used as a bridge which could
allow the AGI research community to import certain abstract results
about mechanical knowing agents from mathematical logic.
\keywords{AGI \and machine knowledge}
\end{abstract}

\section{Introduction}

It is difficult to define knowledge, or what it means to know something.
In Plato's dialogs, again and again Socrates asks people to define
knowledge\footnote{Perhaps the best example being in the \emph{Theaetetus}
\cite{theaetetus}.}, and no-one ever succeeds. Neither have philosophers
reached consensus even in our own era (see \cite{sep-knowledge-analysis}).

At the same time, the problem is often brushed aside as something only
philosophers care about: pragmatists rarely spend time
on this sort of debate. One exceptional area where the question becomes
important even to pragmatists is in the area of Artificial General Intelligence
(AGI). In AGI research, these old philosophical conundrums rear their ugly heads
once more.

In this paper, we narrow down the question ``what is knowledge'' and offer
a simple answer within that narrow context:
we propose a definition of what it means for a suitably idealized AGI to know
a mathematical sentence\footnote{By a \emph{sentence}, we mean a formula with
no free variables. Thus, $x^2>0$ is not a sentence, but
$\forall x (x^2>0)$ is.}. Our proposed definition is short and
sweet: we say that
an AGI knows a mathematical sentence (in some standard mathematical language $\mathscr L$)
if and only if that sentence would be among the sentences which that AGI would
enumerate if that AGI were commanded:
``Enumerate, in the language $\mathscr L$, every mathematical sentence which
is both expressible in $\mathscr L$ and part of your own mathematical knowledge.''

Our proposed definition is not directly intended
for methodological purposes---it would not
directly be helpful in the quest to construct an AGI. Instead, it is intended for
the purpose of understanding the properties of AGI (some of which
we discuss in Section \ref{appsection}). We hope that a better
understanding of theoretical properties of AGIs will indirectly help in the eventual
creation of same.

The structure of this paper is as follows.
\begin{itemize}
  \item In Section \ref{agisection} we discuss the AGIs for whose knowledge we are
  attempting to propose a definition.
  \item In Section \ref{mainsection} we propose a knowledge definition for
  AGIs for sentences in some standard mathematical language.
  \item In Section \ref{quantifiedsection} we extend our knowledge definition
  to formulas with free variables (when the
  language in question is arithmetical).
  \item In Section \ref{appsection} we use this knowledge definition as a bridge
  to allow some results from mathematical logic to inform AGI.
  \item In Section \ref{conclusionsection} we summarize and make concluding remarks.
\end{itemize}

\section{Idealized AGIs}
\label{agisection}



In this paper, we approach AGI using what
Goertzel \cite{goertzel2014artificial} calls
the Universalist Approach:
we adopt ``...an idealized case of AGI, similar to
assumptions like the frictionless plane in physics'', hoping that by
understanding this ``simplified special
case, we can use the understanding we've gained to address more realistic
cases.'' We assume the AGI is designed in such a way as to understand
English commands, and is capable (if so commanded) of outputting data in
any desired format, provided that format can be expressed (say) as a computer
file or as a string of unicode characters. Below, we will state some additional
idealized assumptions, but first we will need a preliminary definition.

\begin{definition}
\label{stdmathematicallanguagedefn}
By a \emph{standard mathematical language}, we mean a mathematical language for which an
intended interpretation is implicitly understood.
\end{definition}

For example, the
language of Peano Arithmetic is a standard mathematical language, the obvious
intended interpretation being the model with universe $\mathbb N$ which
interprets the arithmetical symbols of Peano Arithmetic in the usual ways.
Presumably an AGI suitably familiar with the mathematical literature should
be aware of this intended interpretation for Peano Arithmetic.

Armed with Definition \ref{stdmathematicallanguagedefn}, we are ready to articulate
the key additional assumption which we make about an AGI.

\begin{definition}
\label{obedientdefn}
  An AGI $X$ is \emph{obedient} if, for every standard mathematical language
  $\mathscr L$, the following is true. When commanded to enumerate the
  $\mathscr L$-sentences that he knows, $X$ will obey the command:
  $X$ will enumerate exactly the $\mathscr L$-sentences that he knows
  (according to his own internal definition of knowledge,
  whatever that may be).
\end{definition}


Note that in Definition \ref{obedientdefn}, we do not require the AGI to use
the definition of knowledge that we are proposing in this paper (if we did so,
our definition would be circular).

Definition \ref{obedientdefn} would be inappropriate
for human agents, who are forgetful and error-prone and who tend to resist
tedious tasks such as enumerating endless lists of
mathematical sentences. But arguably Definition \ref{obedientdefn} is plausible for a
sufficiently idealized AGI. Such an AGI can presumably perform calculations with
no risk of mechanical error. And such an AGI should have unlimited patience and have no
problem tediously enumerating mathematical sentences as long as memory-banks
and electricity are available.


\section{An elegant definition of mathematical knowledge}
\label{mainsection}

If $X$ is an idealized AGI which is obedient, we define the mathematical
knowledge of $X$ (as far as sentences go)
as follows (where $\mathscr L$ denotes a standard mathematical
language).

\begin{definition}
\label{maindef}
  For any $\mathscr L$-sentence $\phi$, we say that an
  obedient AGI $X$ knows $\phi$ if and only
  if $X$ would eventually list $\phi$ among the $\mathscr L$-sentences which $X$
  would list if $X$ were commanded:
  ``Enumerate, in the language $\mathscr L$, every mathematical sentence which
  is both expressible in $\mathscr L$ and part of your own mathematical knowledge.''
\end{definition}

One of the strengths of Definition \ref{maindef} is that it is uniform across
different AGIs: many different AGIs might internally operate based on different
definitions of knowledge, but Definition \ref{maindef} works equally well for
all these different AGIs irrespective of those different internal knowledge
definitions\footnote{This is reminiscent of Elton's proposal that instead of
trying to interpret an AI's outputs by focusing on specific low-level details
of a neural network, we should instead let the AI explain itself \cite{elton}.}.

Although Definition \ref{maindef} may differ significantly from a particular AGI
$X$'s own internal definition of knowledge, the following theorem states that
materially the two definitions have the same result.

\begin{theorem}
\label{sentenceequivalence}
  Suppose $X$ is an obedient AGI, and let $\mathscr L$
  be a standard mathematical language. For any $\mathscr L$-sentence $\phi$, the following
  are equivalent:
  \begin{enumerate}
    \item $X$ is
    considered to know $\phi$ (based on Definition \ref{maindef}).
    \item
    $X$ is considered to know $\phi$ (based on $X$'s own inernal definition of
    knowledge).
  \end{enumerate}
\end{theorem}

\begin{proof}
  By Definition \ref{maindef}, (1) is equivalent to the statement that $X$ would
  include $\phi$ in the list which $X$ would output if $X$ were commanded to output
  all the $\mathscr L$-sentences that $X$ knows. Since $X$ is obedient, $X$ would output
  $\phi$ in that list if and only if (2).
\end{proof}

\subsection{Languages with Knowledge Operators}

Definition \ref{maindef} is particularly interesting when $\mathscr L$ itself
contains an operator for the agent's knowledge. An example of such a language would be
the language of Epistemic Arithmetic (or EA) from \cite{shapiro}, which consists of the
language of Peano Arithmetic with the addition of an operator $K$ for knowledge:
$K(1+1=2)$ should be read as something like
``I know $1+1=2$'' or ``the knower knows $1+1=2$''. In the context of this paper,
if $\mathscr L_0$ is a standard mathematical language, and if $\mathscr L$ is obtained
from $\mathscr L_0$ by the addition of a knowledge operator $K$, then we also
consider $\mathscr L$ to be a standard mathematical language. The intended model
of $\mathscr L$ shall have the same universe and interpretation of
$\mathscr L_0$-symbols as the intended model of $\mathscr L_0$. As for $K$,
the intended interpretation (by an AGI $X$) of a formula $K(\phi)$ shall be
that $X$ knows $\phi$ (according to the AGI's internal definition of knowledge).

\begin{example}
Applying Definition \ref{maindef} to the language of EA,
we consider an obedient AGI $X$ to know $K(1+1=2)$ if and only if that AGI would output
$K(1+1=2)$ when commanded to output all sentences that $X$ knows in the language of
EA. By the intended interpretation of
EA, $X$ would (when so commanded)
output $K(1+1=2)$ if and only if $X$ knows that he knows $1+1=2$.
\end{example}

\section{Quantified Modal Logic}
\label{quantifiedsection}

Definition \ref{maindef} only addresses sentences with no free variables.
For suitable languages, we will extend this to formulas which possibly include
free variables. Here, we are essentially adapting a trick from
Carlson \cite{carlson}.

\begin{definition}
  A standard mathematical language $\mathscr L$ is said to be \emph{arithmetical}
  if the following requirements hold.
  \begin{enumerate}
    \item $\mathscr L$ contains all the symbols of Peano Arithmetic.
    \item $\mathscr L$'s intended model has universe $\mathbb N$ and interprets
    the symbols of Peano Arithmetic in the usual ways.
  \end{enumerate}
\end{definition}

\begin{definition}
  If $\mathscr L$ is arithmetical, then we define so-called \emph{numerals}, which
  are $\mathscr L$-terms, one numeral $\overline n$ for each natural number $n\in\mathbb N$,
  by induction: $\overline 0$ is defined to be $0$ (the constant symbol for zero from
  Peano Arithmetic) and
  for every $n\in\mathbb N$, $\overline{n+1}$ is defined to be $S(\overline n)$
  (where $S$ is the successor symbol from Peano Arithmetic).
\end{definition}

For example, the numeral $\overline 5$ is the term $S(S(S(S(S(0)))))$.

\begin{definition}
  If $\mathscr L$ is arithmetical and $\phi$ is an $\mathscr L$-formula (with free variables
  $x_1,\ldots,x_k$),
  and if $s$ is an assignment mapping variables to natural numbers, then we define $\phi^s$
  to be the sentence
  \[
    \phi(x_1|\overline{s(x_1)})(x_2|\overline{s(x_2)})\cdots (x_k|\overline{s(x_k)})
  \]
  obtained by substituting for each free variable $x_i$ the numeral $\overline{s(x_i)}$
  for $x_i$'s value according to $s$.
\end{definition}

\begin{example}
  Suppose $s(x)=0$, $s(y)=1$, and $s(z)=3$. Then
  \[
  ((z>y+x) \wedge \forall x(K(z>y+x-x)))^s
  \]
  is defined to be
  \[
  ((\overline 3 > \overline 1+\overline 0)
  \wedge \forall x( K( \overline 3 > \overline 1 + x - x ) ))
  \]
  (note that the numeral is not substituted for the later occurrences of $x$ because
  these are bound by the $\forall x$ quantifier).
\end{example}

\begin{definition}
\label{maindefextension}
  If $\mathscr L$ is arithmetical, $\phi$ is any $\mathscr L$-formula,
  and $s$ is any assignment mapping variables to $\mathbb N$,
  we say that $X$ knows $\phi$ (according to $s$) if and only if
  $X$ knows $\phi^s$ according to Definition \ref{maindef}.
\end{definition}

Armed with Definition \ref{maindefextension}, the Tarskian notion
\cite{sep-tarski-truth} of
truth can be extended to a complete semantics for
knowledge in any arithmetical language with exactly one knowledge operator $K$.

\begin{example}
  Assume an obedient AGI $X$ is clear from context.
  Suppose $\phi$ is a formula of one free variable $x$, in the language of EA,
  which expresses ``the $x$th Turing machine eventually halts''. Suppose we want to
  assign a truth value to the formula $\exists x (\neg K(\phi)\wedge \neg K(\neg\phi))$.
  \begin{itemize}
  \item Following Tarski, we should declare $\exists x (\neg K(\phi)\wedge\neg K(\neg \phi))$
  is true if and only if there is some assignment $s$ mapping variables to $\mathbb N$
  such that both $K(\phi)$ and $K(\neg\phi)$ are false according to $s$.
  \item By Definition \ref{maindefextension}, this is the case if and only if
  there is some $s$ such that
  $X$ does not know $\phi^s$ and $X$ does not know $\neg\phi^s$
  (according to Definition \ref{maindef}).
  \item
  This is the case if and only if there is some $s$ such that $X$ would not
  list $\phi^s$ nor $\neg\phi^s$ if $X$ were commanded
  to enumerate his own knowledge in the language of EA.
  \item
  Since $\phi$ has just one free variable $x$, it follows that the above is equivalent to:
  there is some $n\in\mathbb N$ such that $X$ would not list $\phi(x|\overline n)$
  nor $\neg\phi(x|\overline n)$
  if $X$ were commanded as above.
  \end{itemize}
\end{example}

\section{Knowledge axioms}
\label{appsection}

In this section, we will look at some axioms of knowledge and interpret them in the
context of AGI in terms of Definitions \ref{maindef} and \ref{maindefextension}.

\begin{example}
  (Basic axioms of knowledge) The following axiom schemas, in the language
  of EA, are taken from Carlson \cite{carlson}
  (we restrict them to sentences for purposes of simplicity).
  \begin{itemize}
    \item (E1) $K(\phi)$ whenever $\phi$ is valid (i.e., a tautology).
    Interpreted for an AGI $X$ using Definition \ref{maindef}, this becomes:
    ``If commanded to enumerate his knowledge in EA, $X$ will include
    all that language's tautologies in the resulting list.'' This is plausible
    because the set of tautologies in a given computable language is computable,
    and an AGI should have no problem enumerating them.
    \item (E2) $K(\phi\rightarrow\psi)\rightarrow K(\phi)\rightarrow K(\psi)$.
    This becomes: ``If commanded to enumerate his knowledge in EA,
    if $X$ would include $\phi\rightarrow\psi$ and if $X$ would also include
    $\phi$, then $X$ would also include $\psi$.'' This is plausible because
    an AGI should certainly be capable of basic logical reasoning.
    \item (E3) $K(\phi)\rightarrow\phi$. This becomes: ``If commanded to enumerate
    his knowledge in EA, the resulting statements $X$ enumerates
    will be true.'' This is plausible since knowledge is widely regarded as
    having truthfulness as one of its requirements. Truthfulness is not a
    requirement in the definition proposed in this paper, but for any particular
    AGI, truthfulness is probably a requirement of that AGI's internal definition
    of knowledge. There is no need to worry about the AGI being misinformed about
    contingent facts about the physical world, because EA is a purely mathematical
    language in which no such contingent facts are expressible.
    \item (E4) $K(\phi)\rightarrow K(K(\phi))$. This becomes: ``If commanded to
    enumerate his knowledge in EA, if $X$ would list $\phi$,
    then $X$ would also list $K(\phi)$.'' This is plausible because presumably
    when $X$ enumerates $\phi$ in response to the command, $X$ should in some sense
    understand why he is enumerating $\phi$, namely because he knows $\phi$---so $X$
    should therefore know that he knows $\phi$, which knowledge is expressible in
    EA as $K(K(\phi))$.
  \end{itemize}
\end{example}

\begin{example}
    (Reinhardt's strong mechanistic thesis
    \cite{reinhardt1985absolute} \cite{reinhardt1986epistemic}
    \cite{carlson}) Reinhardt suggested the
    EA-schema
    \[\exists e \forall x ( K(\phi) \leftrightarrow x\in W_e)\]
    as a formalization of the mechanicalness of the knower. Here, $W_e$
    is the $e$th computably enumerable set of natural numbers ($W_e$ can also
    be thought of as the set of naturals enumerated by the $e$th Turing machine).
    For simplicity, consider the case where $x$ is the lone free variable
    of $\phi$. Then in terms of Definition \ref{maindefextension}, the schema
    becomes:
    ``If $X$ were commanded to enumerate his knowledge in the language of EA,
    then the set of $n\in\mathbb N$ such that $X$ would include $\phi(x|\overline n)$
    in the resulting list, would be computably enumerable.''
    This is not just plausible but obvious\footnote{What is much less obvious
    is the fact that it is consistent for the knower himself to
    know the schema in question. This was conjectured by Reinhardt, and
    proved by Carlson \cite{carlson}. See \cite{aldini2015theory} for some further
    discussion.}, since $X$ himself
    is an AGI and thus presumably a computer.
\end{example}

\begin{example}
  (The Epistemic Church's Thesis \cite{flagg1985church} \cite{carlson2016collapsing})
  The following EA-schema has been suggested as a kind of epistemic formalization
  of Church's Thesis:
  \[
  ( \forall x\exists y (K(\phi))  ) \rightarrow
  ( \exists e K( \forall x\exists y ( E(e,x,y) \wedge \phi  )  )  ),
  \]
  where $E(e,x,y)$ is an EA-formula which expresses that the $e$th Turing machine
  outputs $y$ on input $x$.
  This becomes: ``Suppose $X$ were commanded to enumerate his EA-knowledge.
  Assume there is a (not necessarily computable) function
  $f:\mathbb N\to\mathbb N$ such that for every $n\in\mathbb N$,
  $\phi(x|\overline n)(y|\overline{f(n)})$ is included in the resulting enumeration.
  Then in fact there is a \emph{computable}
  function $f':\mathbb N\to\mathbb N$ (with Turing index $e$)
  such that $f'$ has the same property and such that the enumeration includes
  a statement that $f'$ has said property.''
  This beautiful formalism seems to capture the AGI's self-reflection ability.
  We can imagine the AGI dutifully enumerating statement after statement and
  as she goes, she discovers and predicts patterns in her own enumeration.
  Flagg proved that the Epistemic Church's Thesis is
  consistent with basic axioms of knowledge \cite{flagg1985church},
  and Carlson proved that it is also consistent with
  Reinhardt's strong mechanistic thesis \cite{carlson2016collapsing}.
\end{example}

\section{Conclusion}
\label{conclusionsection}

What does it mean to know something? This is a difficult question and there probably
is no one true answer. In the field of AGI, how can we systematically investigate
the theoretical properties of knowledge, when different AGIs might not even agree
amongst eachother about what knowledge really means? So motivated, we have proposed
an elegant way to brush these philosophical questions aside. In Definition \ref{maindef},
we declare that an AGI knows a sentence in a standard mathematical language if and
only if that AGI would enumerate that sentence if commanded to enumerate all the
things it knows (according to its own internal knowledge definition) in that
mathematical language. In Definition \ref{maindefextension}
we extend this to formulas with free variables, not just sentences.

This one-size-fits-all knowledge definition sets the study of AGI knowledge
on a firmer theoretical footing. In Section \ref{appsection} we give examples
of how our definition can serve as a bridge to translate knowledge-related
results from mathematical logic into the realm of AGI.

\bibliographystyle{splncs04}
\bibliography{short}

\end{document}