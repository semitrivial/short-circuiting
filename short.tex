\documentclass[runningheads]{llncs}
\usepackage{amsfonts}
\usepackage{amsmath}
\usepackage{mathrsfs}
% \smartqed  % flush right qed marks, e.g. at end of proof
% \usepackage{graphicx}

\newcommand{\myclaim}[1]{\textbf{Claim #1:}}
\def\myulcorner{\mathord{\ulcorner}}
\def\myurcorner{\mathord{\urcorner}}

\pagenumbering{gobble}

\begin{document}

\title{Short-circuiting the definition of mathematical knowledge for an AGI
%\thanks{}
}

%\titlerunning{Short-circuiting the definition of knowledge}

\author{Samuel Allen
Alexander\inst{1}\orcidID{0000-0002-7930-110X}}

\institute{The U.S.\ Securities and Exchange Commission
\email{samuelallenalexander@gmail.com}
\url{https://philpeople.org/profiles/samuel-alexander/publications}}


\maketitle

\begin{abstract}
Fill this in.
\keywords{AGI \and machine knowledge \and Moore's paradox}
\end{abstract}

\section{Introduction}

It is difficult to define knowledge, or what it means to know something.
In Plato's dialogs, again and again Socrates asks people to define
knowledge\footnote{Perhaps the best example being in the \emph{Theaetetus}
\cite{theaetetus}.}, and no-one ever succeeds. Neither have philosophers
reached consensus even in our own era (see \cite{sep-knowledge-analysis}).

At the same time, the problem is often brushed aside as something only
philosophers care about: pragmatists rarely spend time
on this sort of debate. One exceptional area where the question becomes
important even to pragmatists is in the area of Artificial General Intelligence
(AGI). In AGI research, these old philosophical conundrums rear their ugly
once more.

In this paper, we narrow down the question ``what is knowledge'' and offer
a simple answer within that narrow context:
we propose a definition of what it means for a suitably idealized AGI to know
a mathematical fact. Our proposed definition is short and sweet: we say that
an AGI knows a mathematical fact (in some standard mathematical language $\mathscr L$)
if and only if that fact would be among the statements which that AGI would
enumerate if that AGI were commanded:
``Enumerate, in the language $\mathscr L$, every mathematical fact which
is both expressible in $\mathscr L$ and part of your own mathematical knowledge.''

Our proposed definition is not directly intended
for methodological purposes---it would not
be helpful in the quest to construct an AGI. Instead, it is intended for
theoretical purposes of understanding the properties of AGI.


\section{Idealized AGIs}

In this paper, we approach AGI using what
Goertzel \cite{goertzel2014artificial} calls
the Universalist Approach:
we adopt ``...an idealized case of AGI, similar to
assumptions like the frictionless plane in physics'', hoping that by
understanding this ``simplified special
case, we can use the understanding we've gained to address more realistic
cases.'' We make the following assumptions about an AGI $X$
(where $\mathscr L$ is some standard mathematical language):
\begin{itemize}
  \item (Truthfulness) Whatever $\mathscr L$-statements $X$ enumerates
  when commanded to enumerate the $\mathscr L$-statements that $X$ knows,
  all such statements are true.
  \item (Obedience) When commanded to enumerate the $\mathscr L$-statements that $X$ knows,
  $X$ will obey the command: $X$ will enumerate exactly the
  $\mathscr L$-statements that $X$ knows (according to $X$'s own internal definition
  of knowledge, whatever that may be).
\end{itemize}

Note that in our \emph{Obedience} assumption, we do not require the AGI to use
the same definition of knowledge that we are proposing in this paper. We allow
the AGI to have whatever definition of knowledge it likes.

The \emph{Truthfulness} and \emph{Obedience} assumptions would be inappropriate
for human agents $X$, who make mathematical
mistakes, and who tend to hold unfounded beliefs and claim them as knowledge,
and who might tend to resist tedious tasks such as enumerating endless lists of
mathematical statements. But arguably these assumptions are plausible for a
sufficiently idealized AGI. Such an AGI can presumably perform calculations with
no risk of mechanical error. Such an AGI is presumably free of the human
psychological quirks which cause humans to cling to unfounded beliefs (especially
since we are only considering mathematical statements, not contingent on
the physical world). And such an AGI should have unlimited patience and have no
problem tediously enumerating mathematical statements as long as electricity is
available.


\section{An elegant definition of mathematical knowledge}

If $X$ is an idealized AGI satisfying the assumptions of \emph{Truthfulness}
and \emph{Obedience} from the previous section, we define the mathematical
knowledge of $X$ as follows (where $\mathscr L$ denotes a standard mathematical
language).

\begin{definition}
\label{maindef}
  For any $\mathscr L$-sentence $\phi$, we say that $X$ knows $\phi$ if and only
  if $X$ would eventually list $\phi$ amount the $\mathscr L$-statements which $X$
  would list if $X$ were commanded:
  ``Enumerate, in the language $\mathscr L$, every mathematical fact which
  is both expressible in $\mathscr L$ and part of your own mathematical knowledge.''
\end{definition}

One of the strengths of Definition \ref{maindef} is that it is uniform across
different AGIs: many different AGIs might internally operate based on different
definitions of knowledge, but Definition \ref{maindef} works equally well for
all these different AGIs irrespective of those different internal knowledge
definitions\footnote{This is reminiscent of Elton's proposal that instead of
trying to interpret an AI's outputs by focusing on specific low-level details
of a neural network, we should instead let the AI explain itself \cite{elton}.}.

\subsection{Languages including knowledge operators}

In this context of AGI knowledge, we are particularly interested in languages
which contain their own operators for knowledge. For example, the language of
Epistemic Arithmetic \cite{shapiro} is identical to the language of Peano
Arithmetic, except for the addition of an operator $K$ for knowledge. Whereas
the formula $1+1=2$ is read as ``one plus one equals two'', the formula
$K(1+1=2)$ might be read as ``I know $1+1=2$'', or ``the knower knows $1+1=2$''.

Assume $X$ is an AGI and $\mathscr L$ is a standard mathematical
language which includes a knowledge operator $K$. According to Definition
\ref{maindef}, we consider $X$ to know $1+1=2$ precisely if $X$ would (if commanded
to list all the $\mathscr L$-statements that $X$ knows) list $1+1=2$.
Likewise, we consider $X$ to know $K(1+1=2)$ precisely if $X$ would (if so commanded)
include $K(1+1=2)$ in said list.

Since we are offering (in Definition \ref{maindef}) a definition for knowledge
which almost surely differs drastically from how the AGI $X$ itself defines knowledge,
the question immediately arises as to whether the two competing definitions (ours and
$X$'s) materially disagree: is there a mathematical fact $\phi$ such that
Definition \ref{maindef} would claim that $X$ knows $\phi$ but such that $X$ itself
would claim that $X$ does not know $\phi$---or vice versa? We will show that if $X$
satisfies a certain very reasonable assumption, then the answer is ``no'': in spite
of whatever metaphysical differences may exist between Definition \ref{maindef} and
$X$'s own definition of knowledge, the two yield the same material results.

\begin{definition}
\label{standardway}
  An AGI $X$ is said to \emph{interpret its own knowledge in a standard way} if
  the following statement holds.
  \begin{itemize}
    \item
    For every standard mathematical language $\mathscr L$
    including a knowledge operator $K$, for every $\mathscr L$-sentence $\phi$,
    if $X$ were commanded to enumerate all the $\mathscr L$-sentences that $X$
    knows, then $X$ would include $\phi$ in that list if and only if $X$ would
    include $K(\phi)$ in that list.
  \end{itemize}
\end{definition}

Definition \ref{standardway} is manifestly reasonable, at least assuming $X$ satisfies
\emph{Truthfulness} and \emph{Obedience}. Suppose $X$ is commanded to enumerate all
the $\mathscr L$-sentences that $X$ knows. If $X$ includes $\phi$ in the resulting
list, it evidently should be because $X$ deems $\phi$ to be among the $\mathscr L$-sentences
that $X$ knows. In that case, apparently $X$ knows that $X$ knows $\phi$, i.e., $X$
knows $K(\phi)$, and so $X$ should also include $K(\phi)$ in said list. Conversely,
if $X$ includes $K(\phi)$ in said list, this evidently should be because $X$ deems
$K(\phi)$ to be among the $\mathscr L$-sentences that $X$ knows. In that case, since
the intended semantics of $K(\phi)$ are literally that $X$ knows $\phi$, clearly (by
Obedience) $X$ should also include $\phi$ itself in said list.

\bibliographystyle{splncs04}
\bibliography{short}

\end{document}