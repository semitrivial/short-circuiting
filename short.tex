\documentclass[runningheads]{llncs}
\usepackage{amsfonts}
\usepackage{amsmath}
\usepackage{mathrsfs}
% \smartqed  % flush right qed marks, e.g. at end of proof
% \usepackage{graphicx}

\newcommand{\myclaim}[1]{\textbf{Claim #1:}}
\def\myulcorner{\mathord{\ulcorner}}
\def\myurcorner{\mathord{\urcorner}}

\pagenumbering{gobble}

\begin{document}

\title{Short-circuiting the definition of mathematical knowledge for an AGI
%\thanks{}
}

%\titlerunning{Short-circuiting the definition of knowledge}

\author{Samuel Allen
Alexander\inst{1}\orcidID{0000-0002-7930-110X}}

\institute{The U.S.\ Securities and Exchange Commission
\email{samuelallenalexander@gmail.com}
\url{https://philpeople.org/profiles/samuel-alexander/publications}}


\maketitle

\begin{abstract}
We propose that, for the purpose of studying theoretical properties of
the knowledge of an agent with Artificial General Intelligence (AGI),
a pragmatic way to define such an agent's knowledge is as follows.
We define that we declare the AGI to know a certain statement
(in a given mathematical language) if and only if, when commanded
to enumerate all the statements that it knows in that language
(with whatever internal knowledge definition it has), said AGI
would include said statement in the resulting enumeration.
This elegantly solves the problem that different AGIs may have
very different internal knowledge definitions and yet we want
to be able to study knowledge of AGIs in general, without having
to study different AGIs separately just because they have separate
internal knowledge definitions. Finally, we suggest how this
definition of AGI knowledge can be used as a bridge which will
allow the AGI research community to import certain abstract results
about mechanical knowing agents from mathematical logic.
\keywords{AGI \and machine knowledge}
\end{abstract}

\section{Introduction}

It is difficult to define knowledge, or what it means to know something.
In Plato's dialogs, again and again Socrates asks people to define
knowledge\footnote{Perhaps the best example being in the \emph{Theaetetus}
\cite{theaetetus}.}, and no-one ever succeeds. Neither have philosophers
reached consensus even in our own era (see \cite{sep-knowledge-analysis}).

At the same time, the problem is often brushed aside as something only
philosophers care about: pragmatists rarely spend time
on this sort of debate. One exceptional area where the question becomes
important even to pragmatists is in the area of Artificial General Intelligence
(AGI). In AGI research, these old philosophical conundrums rear their ugly
once more.

In this paper, we narrow down the question ``what is knowledge'' and offer
a simple answer within that narrow context:
we propose a definition of what it means for a suitably idealized AGI to know
a mathematical sentence\footnote{By a \emph{sentence}, we mean a formula with
no free variables. Thus, $x^2>0$ is not a sentence, but
$\forall x (x^2>0)$.}. Our proposed definition is short and
sweet: we say that
an AGI knows a mathematical sentence (in some standard mathematical language $\mathscr L$)
if and only if that sentence would be among the sentences which that AGI would
enumerate if that AGI were commanded:
``Enumerate, in the language $\mathscr L$, every mathematical sentence which
is both expressible in $\mathscr L$ and part of your own mathematical knowledge.''

Our proposed definition is not directly intended
for methodological purposes---it would not
directly be helpful in the quest to construct an AGI. Instead, it is intended for
theoretical purposes of understanding the properties of AGI (some of which
we survey in Section \ref{appsection}). We hope that a better
understanding of theoretical properties of AGIs will indirectly help in the eventual
creation of same.

The structure of this paper is as follows.
\begin{itemize}
  \item In Section \ref{agisection} we discuss the AGIs whose knowledge we are
  attempting to propose a definition for.
  \item In Section \ref{mainsection} we propose a knowledge definition for
  AGIs for sentences in some standard mathematical language.
  \item In Section \ref{quantifiedsection} we extend our knowledge definition
  to formulas with free variables (when the intended interpretation for the
  language in question is arithmetical).
  \item In Section \ref{appsection} we survey some of the applications of
  this proposed knowledge definition.
  \item In Section \ref{conclusionsection} we summarize and make concluding remarks.
\end{itemize}

\section{Idealized AGIs}
\label{agisection}



In this paper, we approach AGI using what
Goertzel \cite{goertzel2014artificial} calls
the Universalist Approach:
we adopt ``...an idealized case of AGI, similar to
assumptions like the frictionless plane in physics'', hoping that by
understanding this ``simplified special
case, we can use the understanding we've gained to address more realistic
cases.'' We assume the AGI is designed in such a way as to be obey arbitrary
English commands, and is capable (if so commanded) of outputting data in
any desired format, provided that format can be expressed (say) as a computer
file or as a string of unicode characters. Below, we will state some additional
idealized assumptions, but first we will need a preliminary definition.

\begin{definition}
\label{stdmathematicallanguagedefn}
By a \emph{standard mathematical language}, we mean a mathematical language for which an
intended interpretation is implicitly understood.
\end{definition}

For example, the
language of Peano Arithmetic is a standard mathematical language, the obvious
intended interpretation being the model with universe $\mathbb N$ which
interprets the arithmetical symbols of Peano Arithmetic in the usual ways.
Presumably an AGI suitably familiar with the mathematical literature should
be aware of this intended interpretation for Peano Arithmetic.

Armed with Definition \ref{stdmathematicallanguagedefn}, we are ready to articulate
two additional assumptions which we make about an AGI $X$
(here $\mathscr L$ is a standard mathematical language):
\begin{itemize}
  \item (Truthfulness) Whatever $\mathscr L$-sentences $X$ enumerates
  when commanded to enumerate the $\mathscr L$-sentences that $X$ knows,
  all such sentences are true (in the intended model).
  \item (Obedience) When commanded to enumerate the $\mathscr L$-sentences that $X$ knows,
  $X$ will obey the command: $X$ will enumerate exactly the
  $\mathscr L$-sentences that $X$ knows (according to $X$'s own internal definition
  of knowledge, whatever that may be).
\end{itemize}

Note that in our \emph{Obedience} assumption, we do not require the AGI to use
the same definition of knowledge that we are proposing in this paper. We allow
the AGI to have whatever definition of knowledge it likes.

The \emph{Truthfulness} and \emph{Obedience} assumptions would be inappropriate
for human agents $X$, who make mathematical
mistakes, and who tend to hold unfounded beliefs and claim them as knowledge,
and who might tend to resist tedious tasks such as enumerating endless lists of
mathematical sentences. But arguably these assumptions are plausible for a
sufficiently idealized AGI. Such an AGI can presumably perform calculations with
no risk of mechanical error. Such an AGI is presumably free of the human
psychological quirks which cause humans to cling to unfounded beliefs (especially
since we are only considering mathematical statements, not contingent on
the physical world). And such an AGI should have unlimited patience and have no
problem tediously enumerating mathematical sentences as long as memory-banks
and electricity are available.


\section{An elegant definition of mathematical knowledge}
\label{mainsection}

If $X$ is an idealized AGI satisfying the assumptions of \emph{Truthfulness}
and \emph{Obedience} from the previous section, we define the mathematical
knowledge of $X$ (as far as sentences go)
as follows (where $\mathscr L$ denotes a standard mathematical
language).

\begin{definition}
\label{maindef}
  For any $\mathscr L$-sentence $\phi$, we say that $X$ knows $\phi$ if and only
  if $X$ would eventually list $\phi$ among the $\mathscr L$-sentences which $X$
  would list if $X$ were commanded:
  ``Enumerate, in the language $\mathscr L$, every mathematical sentence which
  is both expressible in $\mathscr L$ and part of your own mathematical knowledge.''
\end{definition}

One of the strengths of Definition \ref{maindef} is that it is uniform across
different AGIs: many different AGIs might internally operate based on different
definitions of knowledge, but Definition \ref{maindef} works equally well for
all these different AGIs irrespective of those different internal knowledge
definitions\footnote{This is reminiscent of Elton's proposal that instead of
trying to interpret an AI's outputs by focusing on specific low-level details
of a neural network, we should instead let the AI explain itself \cite{elton}.}.

Although Definition \ref{maindef} may differ significantly from a particular AGI
$X$'s own internal definition of knowledge, the following theorem states that
materially the two definitions have the same result.

\begin{theorem}
\label{sentenceequivalence}
  Suppose $X$ is an AGI satisfying Truthfulness and Obedience, and let $\mathscr L$
  be a standard mathematical language. For any $\mathscr L$-sentence $\phi$, the following
  are equivalent:
  \begin{enumerate}
    \item $X$ is
    considered to know $\phi$ (based on Definition \ref{maindef}).
    \item
    $X$ is considered to know $\phi$ (based on $X$'s own inernal definition of
    knowledge).
  \end{enumerate}
\end{theorem}

\begin{proof}
  By Definition \ref{maindef}, (1) is equivalent to the statement that $X$ would
  include $\phi$ in the list which $X$ would output if $X$ were commanded to output
  all the $\mathscr L$-sentences that $X$ knows. By Obedience, $X$ would output
  $\phi$ in that list if and only if (2).
\end{proof}

\subsection{Languages with Knowledge Operators}

Definition \ref{maindef} is particularly interesting when $\mathscr L$ itself
contains an operator for the agent's knowledge. An example of such a language would be
the language of Epistemic Arithmetic from \cite{shapiro}, which consists of the
language of Peano Arithmetic with the addition of an operator $K$ for knowledge:
$K(1+1=2)$ should be read as something like
``I know $1+1=2$'' or ``the knower knows $1+1=2$''. In the context of this paper,
if $\mathscr L_0$ is a standard mathematical language, and if $\mathscr L$ is obtained
from $\mathscr L_0$ by the addition of a knowledge operator $K$, then we also
consider $\mathscr L$ to be a standard mathematical language. The intended model
of $\mathscr L$ shall have the same universe and interpretation of
$\mathscr L_0$-symbols as the intended model of $\mathscr L_0$. As for $K$,
the intended interpretation (by an AGI $X$) of a formula $K(\phi)$ shall be
that $X$ knows $\phi$ (according to the AGI's internal definition of knowledge).

\begin{example}
Applying Definition \ref{maindef} to the language of Epistemic Arithmetic,
we consider an AGI $X$ to know $K(1+1=2)$ if and only if that AGI would output
$K(1+1=2)$ when commanded to output all sentences that $X$ knows in the language of
Epistemic Arithmetic. By the Obedience assumption and the intended interpretation of
Epistemic Arithmetic, $X$ would (when so commanded)
output $K(1+1=2)$ if and only if $X$ knows that he knows $1+1=2$.
\end{example}

\section{Quantified Modal Logic}
\label{quantifiedsection}

Definition \ref{maindef} only addresses sentences with no free variables.
For suitable languages, we will extend this to formulas which possibly include
free variables. Here, we are essentially adapting a trick which is due to
Carlson \cite{carlson}.

\begin{definition}
  A standard mathematical language $\mathscr L$ is said to be \emph{arithmetical}
  if the following requirements hold.
  \begin{enumerate}
    \item $\mathscr L$ contains all the symbols of Peano Arithmetic.
    \item $\mathscr L$'s intended model has universe $\mathbb N$ and interprets
    the symbols of Peano Arithmetic in the usual ways.
  \end{enumerate}
\end{definition}

\begin{definition}
  If $\mathscr L$ is arithmetical, then we define so-called \emph{numerals}, which
  are $\mathscr L$-terms, one numeral $\overline n$ for each natural number $n\in\mathbb N$,
  by induction: $\overline 0$ is defined to be $0$ (the constant symbol for zero from
  Peano Arithmetic) and
  for every $n\in\mathbb N$, $\overline{n+1}$ is defined to be $S(\overline n)$
  (where $S$ is the successor symbol from Peano Arithmetic).
\end{definition}

For example, the numeral $\overline 5$ is the term $S(S(S(S(S(0)))))$.

\begin{definition}
  If $\mathscr L$ is arithmetical and $\phi$ is an $\mathscr L$-formula (with free variables
  $x_1,\ldots,x_k$),
  and if $s$ is an assignment mapping variables to natural numbers, then we define $\phi^s$
  to be the sentence
  \[
    \phi(x_1|\overline{s(x_1)})(x_2|\overline{s(x_2)})\cdots (x_k|\overline{s(x_k)})
  \]
  obtained by substituting for each variable $x_i$ the numeral $\overline{s(x_i)}$
  for $x_i$'s value according to $s$.
\end{definition}

\begin{example}
  Suppose $s(x)=0$, $s(y)=1$, and $s(z)=3$. Then
  \[
  ((z>y+x) \wedge \forall x(K(z>y+x-x)))^s
  \]
  is defined to be
  \[
  ((\overline 3 > \overline 1+\overline 0)
  \wedge \forall x( K( \overline 3 > \overline 1 + x - x ) ))
  \]
  (note that the numeral is not substituted for the later occurrences of $x$ because
  these are bound by the $\forall x$ quantifier).
\end{example}

\begin{definition}
\label{maindefextension}
  If $\mathscr L$ is arithmetical, $\phi$ is any $\mathscr L$-formula,
  and $s$ is any assignment mapping variables to $\mathbb N$,
  we say that $X$ knows $\phi$ (according to $s$) if and only if,
  $X$ knows $\phi^s$ according to Definition \ref{maindef}.
\end{definition}

Armed with Definition \ref{maindefextension}, the Tarskian notion
\cite{sep-tarski-truth} of
truth can be extended to a complete semantics for
knowledge in any arithmetical language with exactly one knowledge operator $K$.

\begin{example}
  Assume an AGI $X$ is clear from context.
  Suppose $\phi$ is a formula of one free variable $x$, in the language of Epistemic Arithmetic,
  which expresses ``the $x$th Turing machine eventually halts''. Suppose we want to
  assign a truth value to the formula $\exists x (\neg K(\phi))$.
  \begin{itemize}
  \item Following Tarski, we should declare $\exists x (\neg K(\phi))$
  is true if and only if there is some assignment $s$ mapping variables to $\mathbb N$
  such that $K(\phi)$ is false according to $s$.
  \item By Definition \ref{maindefextension}, this is the case if and only if
  there is some $s$ such that
  $X$ does not know $\phi^s$ (according to Definition \ref{maindef}).
  \item
  This is the case if and only if there is some $s$ such that $X$ would not
  list $\phi^s$ if $X$ were commanded
  to enumerate his own knowledge in the language of Epistemic Arithmetic.
  \item
  Since $\phi$ has just one free variable $x$, it follows that the above is equivalent to:
  there is some $n\in\mathbb N$ such that $X$ would not list $\phi(x|\overline n)$
  if $X$ were commanded as above.
  \end{itemize}
\end{example}

\section{Applications}
\label{appsection}

In this section, we survey some results from mathematical logic about the nature
of AGI knowledge. In the literature, these are usually phrased in terms of
formal logical theories: for the purpose of proving things about agent knowledge,
knowing agents are identified with their own knowledge-sets,
with something like Definitions \ref{maindef} and \ref{maindefextension} understood
implicitly.

Here we will briefly and informally summarize some of these mathematical
logical results, glossing them in English in terms of an AGI; for full details,
the reader should read the indicated references. Definitions \ref{maindef}
and \ref{maindefextension} serve as the dictionary which enables these mathematical
logical results to be understood in an AGI context.

\begin{example}
If the reader
reads one of the below references, they might see that the author in question
assumes that knowledge satisfies an axiom schema such as
\[K(\phi\rightarrow\psi)\rightarrow K(\phi)\rightarrow K(\psi).\]
Interpreting this schema using Definition \ref{maindef}, it becomes the following
assumption about AGI $X$ in a relevant language $\mathscr L$:
\begin{itemize}
  \item
  If, when commanded to enumerate all the $\mathscr L$-sentences that he knows,
  $X$ would include $\phi\rightarrow\psi$ in the resulting list, and would also
  include $\phi$ in the resulting list, then $X$ would also include $\psi$ in
  the resulting list.
\end{itemize}
The reader can readily judge the plausibility of this assumption: it is very plausible,
since an AGI should certainly be capable of basic logical reasoning.
\end{example}

\begin{example}
  If one of the below references assumes that knowledge satisfies an axiom schema
  such as
  \[
    K(\phi)\rightarrow K(K(\phi)),
  \]
  then, interpreting this schema using Definition \ref{maindef}, it becomes
  the following assumption about AGI $X$ in a relevant language $\mathscr L$:
  \begin{itemize}
    \item
    If, when commanded to enumerate all the $\mathscr L$-sentences that he knows,
    $X$ would include $\phi$ in the resulting list, then $X$ would also include
    $K(\phi)$ in the resulting list.
  \end{itemize}
  This is plausible because if $X$ performs the action of including $\phi$ in the
  list of $\mathscr L$-sentences he knows, then $X$ should \emph{know} that he
  performs that action, and consequently, since $X$ is an AGI fully capable of
  human (or better) level reasoning, $X$ should certainly therefore know that $X$
  knows $\phi$.
\end{example}

\subsection{Absolute versions of incompleteness theorems}

In his paper \cite{reinhardt1985absolute} about absolute versions of
G\"odel's incompleteness theorems, one of the results can be glossed as follows:
\begin{itemize}
  \item
  If an AGI knows the truthfulness of his own knowledge, then he cannot
  know which Turing machine he is.
\end{itemize}
This was done by finding a specific formula $\phi$ with one free variable $x$,
in the language of Epistemic Arithmetic,
such that
\[
  (*) \,\,\,\,\, \exists e K(\forall x (K(\phi)\leftrightarrow x\in W_e))
\]
is inconsistent with some basic assumptions of knowledge
(including the assumption that $X$ knows the truthfulness of his own knowledge).
Here, $W_e$ denotes the computably enumerable set generated by Turing machine $e$. Thus,
Definition \ref{maindefextension} translates the offending sentence ($*$) into the
following form
for an AGI $X$ (with $\mathscr L$ being the language of Epistemic Arithmetic):
\begin{itemize}
  \item
  $(*) \,\,\,\,\, $There exists some $e$ such that if $X$ were commanded to enumerate his own
  $\mathscr L$-knowledge, then the resulting list would include a sentence
  stating that the $e$th Turing machine generates exactly the list of $n$ such
  that $X$ knows $\phi(x|\overline n)$.
\end{itemize}

In light of Church's Thesis (which asserts that Turing machines capture all
effectively computable algorithms), we can remove the reference to Turing machines
from the gloss of Reinhardt's result and state it in a form even more useful
in the field of AGI research:
\begin{itemize}
  \item
  An AGI cannot know its own truthfulness and its own source-code.
\end{itemize}

Often (including in Reinhardt's own paper), it is taken for granted that
a machine (or an AGI) knows its own truthfulness.
Under that assumption, Reinhardt's result would further simplify to
simply: ``An AGI cannot know its own source-code''.

\subsection{Avoiding incompleteness theorems}

Reinhardt conjectured that although a machine (implicitly, a machine which
knows its own truthfulness) cannot know which Turing machine he is, he can
nevertheless know that he is \emph{some} Turing machine, without knowing
which one. In other words, such a machine can know his own mechanicalness.
This conjecture, now known as \emph{Reinhardt's strong mechanistic thesis}, was
proved by Carlson \cite{carlson} using sophisticated theorems about the
arithmetic of ordinal numbers \cite{carlson1999}\footnote{See
\cite{alexander2015fast} for an elementary proof of a weakened version of
this result.}. Stated in terms of AGI: an AGI can know its own truthfulness
and its own mechanicalness (without knowing exactly which machine it is).

In \cite{alexander2014machine} we showed that if the assumption that the
machine knows its own truthfulness is relaxed (while still requiring that
the machine \emph{be} true, just not necessarily that it knows so),
then it \emph{is} possible for such a machine to exactly know which
Turing machine it is. See also the discussion in \cite{aldini2015theory}.

% \subsection{Generic knowledge}

% \cite{alexander2015fast}

\subsection{Formalizations of Church's Thesis}

Church's Thesis is generally understood as an informal statement that cannot
be formally proven (similar to a physical law of nature).
But in the language of Epistemic Arithmetic, it can be formalized into an
axiom which may or may not hold for particular knowing agents. This formalization
is called the Epistemic Church's Thesis. Written in the language of Epistemic
Arithmetic, it takes the form
\[
  K( ( \forall x\exists y (K(\phi))  ) \rightarrow
  ( \exists e K( \forall x\exists y ( E(e,x,y) \wedge \phi  )  )  )  )
\]
where $E(e,x,y)$ is a formula stating that the $e$th Turing machine
halts on input $x$ with output $y$.
Thus, in English, for an AGI $X$,
the Epistemic Church's Thesis can be glossed as follows:
\begin{itemize}
  \item
  $X$ knows the following: if, for every $m$, there is some $n$ such that
  $X$ knows $\phi(x|\overline m)(y|overline n)$, then there exists some
  Turing machine $e$ such that $X$ knows the following:
  for every $m$, if $m$ is input into $e$, then $e$ will output
  some $n$ such that $\phi(x|\overline m)(y|overline n)$.
\end{itemize}
Or to put it much more simply:
\begin{itemize}
  \item
  $X$ knows the following: if for every $m$ there is some $n$ such that
  $X$ knows $\phi(x|\overline m)(y|overline n)$, then $X$ knows a Turing
  machine which outputs such an $n$ for every input $m$.
\end{itemize}
Flagg showed \cite{flagg1985church} that the Epistemic Church's Thesis is
consistent with basic assumptions of knowledge. Carlson strengthened this
result and showed \cite{carlson2016collapsing} that the Epistemic Church's
Thesis and Reinhardt's strong mechanistic thesis are simultaneously
co-consistent with basic assumptions of knowledge.

\subsection{Knowledge-based measures of intelligence}

At the 2019 CIFMA conference, we presented \cite{alexander2019measuring}
a theoretical measure of intelligence based on mathematical knowledge
(see also \cite{alexander2020agi} where we dubbed this intelligence
measure \emph{Intuitive Ordinal Intelligence}).
In short, we defined an intelligence measure where the intelligence of
a mechanical knowing agent is defined to be the supremum of the
ordinal numbers which have codes that the agent knows to be codes
of ordinal numbers. Using Definition \ref{maindef}, this translates
to AGI as follows. Let $\mathscr L$ be the language of Epistemic Arithmetic
along with an additional unary predicate symbol for Kleene's $\mathcal O$
(a standard set of natural number notations for all the computable
ordinals). For any AGI $X$, the Intuitive Ordinal Intelligence is the
supremum of all those ordinals $\alpha$ such that $\alpha$ has a code
$n\in\mathcal O$ such that, if $X$ were commanded to enumerate all the
$\mathscr L$-sentences he knows, $X$ would include the an $\mathscr L$-sentence
stating that $n\in\mathcal O$.


\section{Conclusion}
\label{conclusionsection}

\bibliographystyle{splncs04}
\bibliography{short}

\end{document}