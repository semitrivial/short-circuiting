\documentclass[runningheads]{llncs}
\usepackage{amsfonts}
\usepackage{amsmath}
\usepackage{mathrsfs}
% \smartqed  % flush right qed marks, e.g. at end of proof
% \usepackage{graphicx}

\newcommand{\myclaim}[1]{\textbf{Claim #1:}}
\def\myulcorner{\mathord{\ulcorner}}
\def\myurcorner{\mathord{\urcorner}}

\pagenumbering{gobble}

\begin{document}

\title{Short-circuiting the definition of knowledge for an AGI
%\thanks{}
}

%\titlerunning{Short-circuiting the definition of knowledge}

\author{Samuel Allen
Alexander\inst{1}\orcidID{0000-0002-7930-110X}}

\institute{The U.S.\ Securities and Exchange Commission
\email{samuelallenalexander@gmail.com}
\url{https://philpeople.org/profiles/samuel-alexander/publications}}


\maketitle

\begin{abstract}
Fill this in.
\keywords{AGI \and machine knowledge \and Moore's paradox}
\end{abstract}

\section{Introduction}

It is difficult to define knowledge, or what it means to know something.
In Plato's dialogs, again and again Socrates asks people to define
knowledge\footnote{Perhaps the best example being in the \emph{Theaetetus}
\cite{theaetetus}.}, and no-one ever succeeds, not even the most renowned sophists.



\bibliographystyle{splncs04}
\bibliography{short}

\end{document}